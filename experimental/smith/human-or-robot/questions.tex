\documentclass{article}

\usepackage{xcolor}
\usepackage{textcomp}

\usepackage{listings}
\lstset{%
basicstyle=\scriptsize,%
numbers=left,%
framexleftmargin=2em,
% numberblanklines=false,
float,
frame=tb,%
breaklines=true,%
postbreak=\raisebox{0ex}[0ex][0ex]{\ensuremath{\color{red}\hookrightarrow\space}},%
% red arrow at line breaks
}

% OpenCL listings
%
% From:
% http://gpumodeling.blogspot.com/2011/06/opencl-programs-in-latex-listings.html
\lstdefinelanguage[OpenCL]{C}[ANSI]{C}
{morekeywords={__kernel,kernel,__local,local,__global,global,%
__constant,constant,__private,private,%
char2,char3,char4,char8,char16,%
uchar2,uchar3,uchar4,uchar8,uchar16,%
short2,short3,short4,short8,short16,%
ushort2,ushort3,ushort4,ushort8,ushort16,%
int2,int3,int4,int8,int16,%
uint2,uint3,uint4,uint8,uint16,size_t,%
long2,long3,long4,long8,long16,%
ulong2,ulong3,ulong4,ulong8,ulong16,%
float2,float3,float4,float8,float16,%
image2d_t,image3d_t,sampler_t,event_t,%
bool2,bool3,bool4,bool8,bool16,%
half2,half3,half4,half8,half16,%
quad,quad2,quad3,quad4,quad8,quad16,%
complex,imaginary},%
}%

\begin{document}

  \title{OpenCL kernel tests}
  \author{Chris Cummins}

  \maketitle

  \section{Introduction}

  This document contains 10 samples of OpenCL kernels. Some of the samples are
  taken from code on GitHub, some of the samples have been synthetically
  generated. Your job is to try and deduce for each whether it was written by a
  human or whether it was synthetically generated. The samples have been rewritten
  so that variable and function names are not taken into consideration. Please
  fill in the attached file \texttt{\input{uid}}with your judgement for each and
  send them it back to me. You do not have to justify your decision, and feel free
  to take as much or as little time as you want. I would expect no more than 30
  seconds spent on each. Thank you for participating!

  \section{Code Samples}

  \lstset{language=[OpenCL]C}
  \lstinputlisting[caption=Sample 1]{sample-1.cl}
  \lstinputlisting[caption=Sample 2]{sample-2.cl}
  \lstinputlisting[caption=Sample 3]{sample-3.cl}
  \lstinputlisting[caption=Sample 4]{sample-4.cl}
  \lstinputlisting[caption=Sample 5]{sample-5.cl}
  \lstinputlisting[caption=Sample 6]{sample-6.cl}
  \lstinputlisting[caption=Sample 7]{sample-7.cl}
  \lstinputlisting[caption=Sample 8]{sample-8.cl}
  \lstinputlisting[caption=Sample 9]{sample-9.cl}
  \lstinputlisting[caption=Sample 10]{sample-10.cl}

  \noindent END OF SAMPLES.

\end{document}
