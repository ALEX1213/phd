\section{Introduction}

This chapter evaluates the performance of OmniTune when tasked with
selecting workgroup sizes for SkelCL stencil codes. First I discuss
measurement noise present in the experimental results, and the methods
used to accommodate for it. Then I examine the observed effect that
workgroup size has on the performance of SkelCL stencils. The
effectiveness of each of the autotuning techniques described in the
previous chapters is evaluated using multiple different machine
learning algorithms. The prediction quality of OmniTune is scrutinised
for portability across programs, devices, and datasets.


\subsubsection{Overview of Experimental Results}

The experimental results consist of measured runtimes for a set of
\emph{test cases}, collected using the methodology explained in the
previous chapter. Each test case $\tau_i$ consists of a scenario,
workgroup size pair $\tau_i = (s_i,w_i)$, and is associated with a
\emph{sample} of observed runtimes from multiple runs of the
program. A total of 269813 test cases were evaluated, which represents
an exhaustive enumeration of the workgroup size optimisation space for
429 scenarios. For each scenario, runtimes for an average of 629 (max
7260) unique workgroup sizes were measured. The average sample size of
runtimes for each test case is 83 (min \checkme{33}, total
\checkme{16917118}).


\section{Statistical Soundness}

\begin{figure}
  \begin{subfigure}[h]{.32\textwidth}
  \centering
  \includegraphics{img/runtimes_histogram_1}
  \vspace{-1.5em} % Shrink vertical padding
  \caption{}
  \label{fig:runtimes-histogram-1}
\end{subfigure}
~%
\begin{subfigure}[h]{.32\textwidth}
  \centering
  \includegraphics{img/runtimes_histogram_2}
  \vspace{-1.5em} % Shrink vertical padding
  \caption{}
  \label{fig:runtimes-histogram-2}
\end{subfigure}
~%
\begin{subfigure}[h]{.32\textwidth}
  \centering
  \includegraphics{img/runtimes_histogram_3}
  \vspace{-1.5em} % Shrink vertical padding
  \caption{}
  \label{fig:runtimes-histogram-3}
\end{subfigure}
\\
\begin{subfigure}[h]{.32\textwidth}
  \centering
  \includegraphics{img/runtimes_histogram_4}
  \vspace{-1.5em} % Shrink vertical padding
  \caption{}
  \label{fig:runtimes-histogram-4}
\end{subfigure}
~%
\begin{subfigure}[h]{.32\textwidth}
  \centering
  \includegraphics{img/runtimes_histogram_5}
  \vspace{-1.5em} % Shrink vertical padding
  \caption{}
  \label{fig:runtimes-histogram-5}
\end{subfigure}
~%
\begin{subfigure}[h]{.32\textwidth}
  \centering
  \includegraphics{img/runtimes_histogram_6}
  \vspace{-1.5em} % Shrink vertical padding
  \caption{}
  \label{fig:runtimes-histogram-6}
\end{subfigure}
\\
\begin{subfigure}[h]{.32\textwidth}
  \centering
  \includegraphics{img/runtimes_histogram_7}
  \vspace{-1.5em} % Shrink vertical padding
  \caption{}
  \label{fig:runtimes-histogram-7}
\end{subfigure}
~%
\begin{subfigure}[h]{.32\textwidth}
  \centering
  \includegraphics{img/runtimes_histogram_8}
  \vspace{-1.5em} % Shrink vertical padding
  \caption{}
  \label{fig:runtimes-histogram-8}
\end{subfigure}
~%
\begin{subfigure}[h]{.32\textwidth}
  \centering
  \includegraphics{img/runtimes_histogram_9}
  \vspace{-1.5em} % Shrink vertical padding
  \caption{}
  \label{fig:runtimes-histogram-9}
\end{subfigure}

  \caption[Distribution of stencil code runtimes]{%
  Distribution of runtime samples for test cases from three
  devices. Each plot contains a 35-bin histogram of 1000 samples, and
  a fitted kernel density estimate with bandwidth 0.3. The sample mean
  is shown as a vertical dashed line. The top row are from the Intel
  i5-4570, the second row from the Nvidia GTX 590, and the third row
  from the AMD Tahiti 7970. In some of the plots, the distribution of
  runtimes is bimodal, and skewed to the lower end of the runtimes
  range.%
  }
  \label{fig:runtime-histograms}
\end{figure}

The complex interaction between processes competing for the finite
resources of a system introduces many sources for noise in program
runtime measurements. Before making any judgements about the relative
performance of optimisation configurations, we must establish the
level of noise present in these measurements. To do this, we evaluate
the distribution of runtimes for a randomly selected 1000 test cases,
recording 1000 runtime observations for each. We can then produce
fine-grained histograms of runtimes for individual test
cases. Figure~\ref{fig:runtime-histograms} shows an example nine of
these, for test cases from three devices. The plots show that the
distribution of runtimes is not always Gaussian; rather, it is
sometimes bimodal, and generally skewed to the lower end of the
runtime range, with a long ``tail'' to the right. This fits our
intuition that programs have a hard \emph{minimum} runtime enforced by
the time taken to execute the instructions of a program, and that
noise introduced to the system extends this runtime. For example,
preempting an OpenCL process on a CPU so that another process may run
may cause the very long tail visible in
Figure~\ref{fig:runtimes-histogram-1}.

The central limit theorem allows the assumption of an underlying
Gaussian distribution for samples of size $\ge 30$~\cite{Georges2007}.
Given our minimum sample size of 33, we can use 95\% confidence
intervals to provide statistical confidence that the arithmetic mean
of observed runtimes with respect to the true mean. As the number or
samples increases, we should expect the size of the confidence
interval to shrink. This is illustrated in Figure~\ref{fig:ci-trends},
which plots the average size of 95\% confidence intervals across the
1000 test cases, normalised to their respective means, as a function
of sample size. It shows the diminishing returns that increasing
sample size provides. For example, increasing the sample count from 10
to 30 results in an approximate 50\% reduction in confidence interval
size. Increasing the sample size from 30 to 50 results in only a 25\%
reduction.

\begin{figure}
  \centering
  \includegraphics{img/ci_trend}
  \caption[Confidence interval size vs.\ sample count]{%
  Ratio of confidence interval to mean as a function of sample
  count. Two dashed lines indicate the confidence intervals at the
  minimum (3.7\%) and mean (2.5\%) number of samples used in the
  experimental dataset.%
  }
  \label{fig:ci-trends}
\end{figure}

By comparing the average confidence interval at different sample
counts against the full experiment results of 269813 test cases, we
can assert with 95\% confidence that the true mean for each test case
is within 2.5\% of the sample mean (given the average number of
samples per test case), or 3.7\% in the worst case (at the minimum
number of samples). Since the differences between baseline and optimal
workgroup sizes is often well in excess of 100\%, there is no overlap
of confidence intervals between competing workgroup sizes.
% \FIXME{This demonstrates a sufficiently low level of noise that
% meaningful comparisons can be made between the performance of
% different configurations.}


\section{Workgroup Size Optimisation Space}

In this section we explore the impact that the workgroup size
optimisation space has on the performance of stencil codes.

\subsection{Oracle Workgroup Sizes}

\begin{figure}
  \centering
  \includegraphics{img/num_params_oracle.pdf}
  \caption[Oracle accuracy vs.\ number of workgroup sizes]{%
  Accuracy compared to the oracle as a function of the number of
  workgroup sizes used. The best accuracy that can be achieved using a
  single statically chosen workgroup size is 15\%. Achieving 50\%
  oracle accuracy requires a minimum of 14 distinct workgroup sizes.%
  }
  \label{fig:oracle-accuracy}
\end{figure}

For each scenario $s$, the oracle workgroup size $\Omega(s)$ is the
workgroup size which resulted in the lowest mean runtime. If the
performance of stencils were independent of workgroup size, we would
expect that the oracle workgroup size would remain constant across all
scenarios $s \in S$. Instead, we find that there are 135 unique oracle
workgroup sizes, with 31.5\% of scenarios having a unique workgroup
size. This demonstrates the difficult in attempting to tune for
\emph{optimal} parameter values, since 14 distinct workgroup sizes are
needed to achieve just 50\% of the oracle accuracy
(Figure~\ref{fig:oracle-accuracy}), although it is important to make
the distinction that oracle \emph{accuracy} and \emph{performance} are
not equivalent.

Figure~\ref{fig:oracle-wgsizes} shows the distribution of oracle
workgroup sizes, demonstrating that there is clearly no ``silver
bullet'' workgroup size which is optimal for all scenarios, and that
the space of oracle workgroup sizes is non linear and complex. The
workgroup size which is most frequently optimal is
$w_{(64 \times 4)}$, which is optimal for 15\% of scenarios. Note that
this is not adequate to use as a baseline for static tuning, as it
does not respect legality constraints, that is
$w_{(64 \times 4)} \not\in W_{safe}$.


\subsection{Workgroup Size Legality}\label{subsec:legality}

\begin{figure}
  \begin{subfigure}[t]{0.98\textwidth}
    \centering
    \includegraphics{img/oracle_param_space.pdf}
    \vspace{-1.5em} % Shrink vertical padding
    \caption{}
    \label{fig:oracle-wgsizes}
  \end{subfigure}
  \\
  \begin{subfigure}[t]{0.98\textwidth}
    \centering
    \includegraphics{img/coverage_space.pdf}
    \vspace{-1.5em} % Shrink vertical padding
    \caption{}
    \label{fig:coverage}
  \end{subfigure}
  \caption[Workgroup size legality and optimality]{%
  Log frequency counts for: (\subref{fig:oracle-wgsizes}) optimality,
  and (\subref{fig:coverage}) legality for a subset of the aggregated
  workgroup size optimisation space, $w_c \le 100, w_r \le 100$. The
  space of oracle workgroup size frequencies is highly irregular and
  uneven, with a peak frequency of $w_{(64 \times 4)}$. Legality
  frequencies are highest for smaller row and column counts (where
  $w < W_{\max}(s) \forall s \in S$), and $w_c$ and $w_r$ values which
  are multiples of 8.%
  }
  \label{fig:heatmaps}
\end{figure}

As explained in Section~\ref{sec:op-params}, the space of legal
workgroup sizes $W_{legal}(s)$ for a given scenario $s$ comprises all
workgroup sizes which: do not exceed the maximum allowed by the OpenCL
device and kernel $W_{\max}(s)$, and are not refused by the OpenCL
runtime.

\subsubsection{Maximum workgroup sizes}

\begin{figure}
  \centering
  \includegraphics{img/max_wgsizes.pdf}
  \vspace{-1.5em} % Shrink vertical padding
  \caption[Workgroup size coverage]{%
  A subset of the aggregated workgroup size optimisation space,
  $w_c \le 100, w_r \le 100$, showing the \emph{coverage} of each
  workgroup size, i.e.\ the ratio of scenarios for which a workgroup
  size satisfies architecture and kernel enforced constraints
  ($W_{\max}(s)$). Workgroup sizes with a coverage of $< 1$ fail to
  satisfy these constraints for one or more scenarios. Only
  workgroup sizes with a coverage of 1 may be used for static
  tuning, which greatly reduces the size of the optimisation
  space. Observed $W_{\max}(s)$ values are multiples of 256, hence
  the abrupt ``steps'' in coverage.%
  }
  \label{fig:max-wgsizes}
\end{figure}

We define the \emph{coverage} of a workgroup size to be the ratio
$0 \le x \le 1$ between the number of scenarios for which the
workgroup size was less than $W_{\max}(s)$, normalised to the total
number of workgroup sizes. A coverage of 1 implies a workgroup size
which is always legal for all combinations of stencil and
architecture. A workgroup size with a coverage of 0 is never
legal. Figure~\ref{fig:max-wgsizes} plots the coverage of a subset of
the workgroup size optimisation space.

Note that since $W_{\max}(s)$ defines a hard limit for a given $s$, if
statically selecting a workgroup size, one must limit the optimisation
space to the smallest $W_{\max}(s)$ value, i.e.\ only the workgroup
sizes with a coverage of 1. The observed $W_{\max}(s)$ values range
from 256--8192, which results in up to a 97\% reduction in the size of
the optimisation space when $W_{\max}(s) = 8192$, even though only
14\% of scenarios have the minimum value of $W_{\max}(s) = 256$.

% Size of optimisation space for Wmax =  256: 273
% Size of optimisation space for Wmax = 8192: 15925

\subsubsection{Refused Parameters}

\begin{table}
  \parbox{.32\linewidth}{
  \centering
  \scriptsize
  \rowcolors{2}{white}{gray!25}
  \begin{tabular}{lr}
  \toprule
  Parameter & Refused (\%) \\
  \midrule
  $18 \times 24$ & 0.32 \\
  $26 \times 16$ & 0.31 \\
  $28 \times 32$ & 0.31 \\
  $4 \times 2$ & 0.31 \\
  $16 \times 14$ & 0.30 \\
  $14 \times 40$ & 0.30 \\
  $24 \times 36$ & 0.30 \\
  $36 \times 24$ & 0.30 \\
  $4 \times 26$ & 0.30 \\
  $4 \times 6$ & 0.30 \\
  \bottomrule
\end{tabular}


  }
  \hfill
  \parbox{.32\linewidth}{
  \centering
  \scriptsize
  \rowcolors{2}{white}{gray!25}
  \begin{tabular}{lr}
  \toprule
  Parameter & Refused (\%) \\
  \midrule
  $8 \times 18$ & 0.30 \\
  $8 \times 22$ & 0.30 \\
  $14 \times 48$ & 0.29 \\
  $28 \times 24$ & 0.29 \\
  $30 \times 8$ & 0.29 \\
  $48 \times 14$ & 0.29 \\
  $8 \times 10$ & 0.29 \\
  $10 \times 8$ & 0.28 \\
  $24 \times 18$ & 0.28 \\
  $26 \times 32$ & 0.28 \\
  \bottomrule
\end{tabular}


  }
  \hfill
  \parbox{.32\linewidth}{
  \centering
  \scriptsize
  \rowcolors{2}{white}{gray!25}
  \begin{tabular}{lr}
  \toprule
  Parameter & Refused (\%) \\
  \midrule
  $26 \times 8$ & 0.28 \\
  $2 \times 32$ & 0.28 \\
  $2 \times 48$ & 0.28 \\
  $4 \times 42$ & 0.28 \\
  $50 \times 4$ & 0.28 \\
  $10 \times 48$ & 0.28 \\
  $12 \times 24$ & 0.28 \\
  $16 \times 18$ & 0.28 \\
  $24 \times 2$ & 0.28 \\
  $48 \times 18$ & 0.28 \\
  \bottomrule
\end{tabular}


  }
  \caption[Workgroup sizes most frequently refused]{%
  The thirty most refused parameters, ranked in descending
  order. There is little correlation between the size of workgroup and the
  likelihood that it is refused, suggesting that the cause of refused
  parameters is not a resource constraint, but a behavioural issue.%
  }
  \label{tab:top-refused-params}
\end{table}

\begin{figure}
  \centering
  \begin{subfigure}[h]{.45\textwidth}
    \centering
    \includegraphics{img/refused_params_by_device}
    \caption{}
    \label{fig:refused-params-by-device}
  \end{subfigure}
  \hfill
  \begin{subfigure}[h]{.45\textwidth}
    \centering
    \includegraphics{img/refused_params_by_vendor}
    \caption{}
    \label{fig:refused-params-by-vendor}
  \end{subfigure}
  \caption[Refused workgroup sizes by device and vendor]{%
  The ratio of test cases with refused workgroup sizes, grouped by:
  (\subref{fig:refused-params-by-device}) OpenCL device ID;\
  (\subref{fig:refused-params-by-vendor}) device vendor ID. Parameters
  were refused most frequently by Intel i5 CPUs, then by
  previous-generation NVIDIA GPUs. No parameters were refused by AMD
  devices.%
  }
  \label{fig:refused-params-by-dev-vendor}
\end{figure}

In addition to the hard constraints imposed by the maximum workgroup
size, there are also refused parameters, which are workgroup sizes
which are rejected by the OpenCL runtime and do not provide a
functioning program. Of the 8504 unique workgroup sizes tested, 11.4\%
were refused in one or more test cases. An average of 5.5\% of all
test cases lead to refused parameters. For a workgroup size to be
refused, it must satisfy the architectural and program-specific
constraints which are exposed by OpenCL, but still lead to a
\texttt{CL\_OUT\_OF\_RESOURCES} error when the kernel is
enqueued. Table~\ref{tab:top-refused-params} lists the most frequently
refused parameters, and the percentage of test cases for which they
were refused. While uncommon, a refused parameter is an obvious
inconvenience to the user, as one would expect that any workgroup size
within the specified maximum should behave \emph{correctly}, if not
efficiently. Figure~\ref{fig:coverage} visualises the space of legal
workgroup sizes by showing the frequency counts that a workgroup size
is legal. Smaller workgroup sizes are legal most frequently due to the
$W_{\max}(s)$ constraints. Beyond that, workgroup sizes which contain
$w_c$ and $w_r$ values which are multiples of eight are more
frequently legal, which is a common width of SIMD vector
operations~\cite{IntelCorporation2012}.

Experimental results suggest that the problem is vendor --- or at
least device --- specific. By grouping the refused test cases by
device and vendor, we see a much greater quantity of refused
parameters for test cases on Intel CPU devices than any other type,
while no workgroup sizes were refused by the AMD
GPU. Figure~\ref{fig:refused-params-by-dev-vendor} shows these
groupings. The exact underlying cause for these refused parameters is
unknown, but can likely by explained by inconsistencies or errors in
specific OpenCL driver implementations.

As these OpenCL implementations are still in active development, it is
anticipated that errors caused by unexpected behaviour will become
more infrequent as the technology
matures. Figure~\ref{fig:refused-params-by-device} shows that the
ratio of refused parameters decreases across the three generations of
Nvidia GPUs: GTX 590 (2011), GTX 690 (2012), and GTX TITAN (2013). The
same trend is apparent for the two Intel i5s: i5-2430M (2011), and
i5-4570 (2013), although not for the i7-3820 (2012). For now, it is
imperative that any autotuning system is capable of adapting to these
refused parameters by suggesting alternatives when they occur.


\subsection{Baseline Parameter}\label{subsec:baseline}

The baseline parameter $\bar{w}$ is the workgroup size which provides
the best overall performance while being legal for all scenarios. It
is the workgroup size $w \in W_{safe}$ which maximises the output of
the performance function $\bar{p}(w)$. As shown in
Table~\ref{tab:highest-legality}, only a \emph{single} workgroup size
$w_{(4 \times 4)}$ from the set of experimental results is found to
have a legality of 100\%, suggesting that an adaptive approach to
setting workgroup size is necessary not just for the sake of
maximising performance, but also for guaranteeing program correctness.

The utility of the baseline parameter is that it represents the best
performance that can be achieved through static tuning of the
workgroup size parameter. We can evaluate the performance of
suboptimal workgroup sizes by calculating the geometric mean of their
\emph{performance} for a particular scenario $p(s, w)$ across all
scenarios, $\bar{p}(w)$. The baseline parameter $\bar{p}(\bar{w})$
achieves only 24\% of the available
performance. Figure~\ref{fig:performance-legality} plots workgroup
size \emph{legality} and \emph{performance}, showing that there is no
clear correlation between the two. In fact, the workgroup sizes with
the highest mean performance are valid only for scenarios with the
largest $W_{\max}(s)$ value, which account for less than 1\% of all
scenarios, further reinforcing the case for adaptive tuning. The
workgroup sizes with the highest legality are listed in
Table~\ref{tab:highest-legality}, and the workgroup sizes with the
highest performance are listed in Table~\ref{tab:highest-performance}.

Figure~\ref{fig:speedups} shows the speedup of the oracle workgroup
size over the baseline parameter $w_{(4 \times 4)}$ for all
scenarios. If we assume that sufficiently pragmatic developer with
enough time would eventually find this static optimal, then this
provides a reasonable comparison for calculating speedups of an
autotuner for workgroup size. Comparing the runtime of workgroup sizes
relative to the oracle allows us to calculate upper bounds on the
possible performance which can be expected from autotuning.


\begin{figure}
  \centering
  \includegraphics{img/params_summary.pdf}
  \caption[Workgroup size legality vs.\ performance]{%
  Average legality and performance relative to the oracle of all
  workgroup sizes. Clearly, the relationship between legality and
  performance is not linear. Distinct vertical ``bands'' appear
  between regions of legality caused by the different $W_{\max}(s)$
  values of devices. The irregular jitter between these vertical bands
  is caused by refused parameters.%
  }
  \label{fig:performance-legality}
\end{figure}


\begin{table}[t]
  \parbox{.45\linewidth}{
  \centering
  \scriptsize
  \rowcolors{2}{white}{gray!25}
  \begin{tabular}{lrr}
  \toprule
  Parameter & Legality (\%) & Performance (\%) \\
  \midrule
  $4 \times 4$ & 100.0 & 34.6 \\
  $32 \times 4$ & 97.4 & 79.2 \\
  $40 \times 4$ & 96.5 & 77.5 \\
  $16 \times 4$ & 96.3 & 65.5 \\
  $56 \times 4$ & 95.6 & 81.0 \\
  $16 \times 8$ & 95.6 & 75.7 \\
  $8 \times 4$ & 95.6 & 51.7 \\
  $24 \times 4$ & 95.3 & 69.6 \\
  $24 \times 8$ & 95.1 & 78.4 \\
  $48 \times 4$ & 94.6 & 83.1 \\
  $8 \times 8$ & 94.6 & 63.4 \\
  $4 \times 16$ & 94.6 & 50.1 \\
  $4 \times 8$ & 94.4 & 41.5 \\
  $8 \times 16$ & 94.2 & 70.6 \\
  $4 \times 40$ & 94.2 & 50.2 \\
  $4 \times 32$ & 93.9 & 53.6 \\
  $4 \times 48$ & 93.9 & 51.9 \\
  $4 \times 56$ & 93.9 & 50.3 \\
  $4 \times 24$ & 93.5 & 50.4 \\
  $8 \times 24$ & 92.5 & 69.5 \\
  $64 \times 4$ & 84.1 & 87.5 \\
  $16 \times 16$ & 84.1 & 74.5 \\
  $24 \times 16$ & 83.2 & 75.1 \\
  $32 \times 8$ & 83.0 & 81.3 \\
  $40 \times 16$ & 82.8 & 80.3 \\
  \bottomrule
\end{tabular}


  \caption[Workgroup sizes with greatest legality]{%
  The 25 workgroup sizes with the greatest legality.%
  }
  \label{tab:highest-legality}
  }
  \hfill
  \parbox{.45\linewidth}{
  \centering
  \scriptsize
  \rowcolors{2}{white}{gray!25}
  \begin{tabular}{lrr}
  \toprule
  Parameter & Legality (\%) & Performance (\%) \\
  \midrule
  $270 \times 24$ & 0.2 & 97.5 \\
  $174 \times 38$ & 0.2 & 94.4 \\
  $310 \times 20$ & 0.2 & 94.2 \\
  $546 \times 10$ & 0.2 & 93.0 \\
  $282 \times 16$ & 0.2 & 93.0 \\
  $520 \times 10$ & 0.2 & 92.9 \\
  $520 \times 12$ & 0.2 & 92.3 \\
  $746 \times 8$ & 0.2 & 92.0 \\
  $38 \times 140$ & 0.2 & 91.9 \\
  $300 \times 18$ & 0.2 & 91.3 \\
  $700 \times 6$ & 0.2 & 91.1 \\
  $96 \times 24$ & 24.7 & 90.6 \\
  $88 \times 48$ & 0.7 & 90.6 \\
  $820 \times 6$ & 0.2 & 90.4 \\
  $88 \times 32$ & 24.9 & 90.0 \\
  $96 \times 32$ & 25.4 & 89.8 \\
  $88 \times 40$ & 21.0 & 89.7 \\
  $80 \times 16$ & 30.3 & 89.6 \\
  $722 \times 10$ & 0.2 & 89.5 \\
  $280 \times 24$ & 0.2 & 89.5 \\
  $910 \times 6$ & 0.2 & 89.4 \\
  $88 \times 24$ & 28.7 & 89.4 \\
  $64 \times 24$ & 29.4 & 89.2 \\
  $80 \times 32$ & 24.5 & 89.2 \\
  $72 \times 16$ & 29.4 & 89.2 \\
  \bottomrule
\end{tabular}


  \caption[Workgroup sizes with greatest performance]{%
  The 25 workgroup sizes with the greatest mean
  performance.%
  }
  \label{tab:highest-performance}
  }
\end{table}


\subsection{Speedup Upper Bounds}

\begin{figure}[t]
  \includegraphics{img/max_speedups}
  \caption[Workgroup size speedups]{%
  Speedup of oracle workgroup size over: the worst performing
  workgroup size for each scenario (\emph{Max}), the statically
  chosen workgroup size that provides the best overall performance
  ($w_{(4 \times 4)}$), and the human expert selected parameter
  ($w_{(32 \times 4)}$). Note that the human expert parameter is not
  legal for all scenarios.%
  }
  \label{fig:speedups}
\end{figure}

For a given scenario $s$, the ratio of the workgroups sizes from
$W_{legal}(s)$ which provide the longest and shortest mean runtimes is
used to calculate an upper bound for the possible performance
influence of workgroup size:
%
\begin{equation}
  r_{max}(s) = r(s, \argmax_{w \in W_{legal}(s)} t(s,w), \Omega(s))
\end{equation}
%
When applied to each scenario $s \in S$ of the experimental results,
we find the average of speedup upper bounds to be $15.14\times$ (min
$1.03\times$, max $207.72\times$). This demonstrates the importance of
tuning stencil workgroup sizes --- if chosen incorrectly, the runtime
of stencil programs can be extended by up to $207.72\times$. Note too
that for 5 of the scenarios, the speedup of the best over worst
workgroup sizes is $\le 5\%$.
% TODO: t-test for this!
For these scenarios, there is little benefit to autotuning; however,
this represents only 1.1\% of the tested scenarios. For 50\% of the
scenarios, the speedup of the best over worst workgroup sizes is
$\ge 6.19\times$.


\subsection{Human Expert}

% SCENARIOS IN WHICH 32x4 WAS NOT LEGAL:
%
% sqlite> select distinct name,kernels.north,kernels.south,kernels.east,kernels.west,device,dataset,name from runtime_stats left join scenarios on runtime_stats.scenario=scenarios.id left join kernel_names on scenarios.kernel=kernel_names.id left join kernels on scenarios.kernel=kernels.id where scenario NOT IN (select scenario from runtime_stats where params="32x4");
% name        north       south       east        west        device                                     dataset                name
% ----------  ----------  ----------  ----------  ----------  -----------------------------------------  ---------------------  ----------
% complex     30          30          30          30          1xIntel(R) Core(TM) i5-4570 CPU @ 3.20GHz  1024.1024.float.float  complex
% simple      0           0           0           0           1xIntel(R) Core(TM) i5-2430M CPU @ 2.40GH  1024.1024.float.float  simple
% complex     30          30          30          30          1xIntel(R) Core(TM) i5-2430M CPU @ 2.40GH  2048.2048.float.float  complex
% complex     1           10          30          30          1xIntel(R) Core(TM) i5-4570 CPU @ 3.20GHz  512.512.float.float    complex
% simple      30          30          30          30          1xIntel(R) Core(TM) i5-2430M CPU @ 2.40GH  512.512.float.float    simple
% complex     30          30          30          30          1xGeForce GTX 690                          512.512.float.float    complex
% simple      20          10          20          10          1xIntel(R) Core(TM) i5-2430M CPU @ 2.40GH  4096.4096.float.float  simple
% simple      1           10          30          30          1xIntel(R) Core(TM) i5-2430M CPU @ 2.40GH  2048.2048.float.float  simple
% complex     30          30          30          30          1xGeForce GTX 690                          1024.1024.float.float  complex
% complex     1           10          30          30          1xIntel(R) Core(TM) i5-2430M CPU @ 2.40GH  2048.2048.float.float  complex
% simple      30          30          30          30          1xGeForce GTX 690                          512.512.float.float    simple

In the original implementation of the SkelCL stencil
skeleton~\cite{Breuer2013}, \citeauthor{Breuer2013} selected a
workgroup size of $w_{(32 \times 4)}$ in an evaluation of 4 stencil
operations on a Tesla S1070 system. We can use this as an additional
parameter to compare the relative performance of workgroup sizes
against. However, the $w_{(32 \times 4)}$ workgroup size is invalid
for 2.6\% of scenarios, as it is refused in 11 test cases. By device,
those are: 3 on the GTX 690, 6 on the i5-2430M, and 2 on the i5-4570.
For the scenarios where $w_{(32 \times 4)}$ \emph{is} legal, the human
expert chosen workgroup size achieves an impressive geometric mean of
79.2\% of the oracle performance. The average speedup of oracle
workgroup sizes over the workgroup size selected by a human expert is
$1.37\times$ (min $1.0\times$, max $5.17\times$). Since the workgroup
size selected by the human expert is not legal for all scenarios, we
will examine the effectiveness of heuristics for tuning workgroup
size.


\subsection{Heuristics}\label{subsec:heuristics}

In this section we will consider the effectiveness of instead
selecting workgroup size using two types of heuristics. The first,
using the maximum workgroup size returned by the OpenCL device and
kernel APIs to select the workgroup size adaptively. The second, using
per-device heuristics, in which the workgroup size is selected based
on the specific architecture that a stencil is operating on.

\subsubsection{Using maximum legal size}

A common approach taken by OpenCL developers is to set the workgroup
size for a kernel based on the maximum legal workgroup size queried
from the OpenCL APIs. For example, to set the size of 2D workgroup, a
developer the square root of the (scalar) maximum wgsize to set the
number of columns and rows (since $w_c \cdot w_r$ must be
$< W_{\max}(s)$). To consider the effectiveness of this approach, we
group the workgroup size performances based on the ratio of the
maximum allowed for each scenario. We can also perform this for each
of the two dimensions --- rows and columns --- of the stencil
workgroup size.

Figure~\ref{fig:performance-wgsizes} shows the distribution of
runtimes when grouped this way, demonstrating that the performance of
(legal) workgroup sizes are not correlated with the maximum workgroup
sizes $W_{\max}(s)$. However, when considering individual components,
we observe that the best median workgroup size performances are
achieved with a number of columns that is between 10\% and 20\% of the
maximum, and a number of rows that is between 0\% and 10\% of the
maximum.

\subsubsection{Per-device workgroup sizes}

\begin{table}
  \scriptsize
  \centering
  \rowcolors{2}{white}{gray!25}
  \begin{tabular}{lllll}
  \toprule
  Device & Oracle & Legality & Perf.\ min & Perf.\ avg. \\
  \midrule
  AMD Tahiti 7970 &   $48\times 4$ & 1.0 & 0.54 & 0.91 \\
  Intel i5-2430M &  $64\times 16$ & 0.8 & 0.37 & 0.91 \\
  Intel i5-4570 &   $88\times 8$ & 0.89 & 0.33 & 0.89 \\
  Intel i7-3820 &  $40\times 24$ & 0.95 & 0.76 & 0.97 \\
  NVIDIA GTX 590 &  $12\times x2$ & 0.8 & 0.2 & 0.9 \\
  NVIDIA GTX 690 &   $64\times 4$ & 0.93 & 0.32 & 0.84 \\
  NVIDIA GTX TITAN &   $64\times 4$ & 1.0 & 0.26 & 0.81 \\
  \textbf{GPUs} &   $64\times 4$ & 0.76 & 0.26 & 0.86 \\
  \textbf{CPUs} &   $88\times 8$ & 0.88 & 0.33 & 0.91 \\
  \bottomrule
\end{tabular}


  \caption[Performance of tuning with a per-device heuristic]{%
  Selecting workgroup size using a per-device heuristic. The mode
  optimal workgroup size for each device type $\bar{w}$ is evaluated
  based on legality, and relative performance to the oracle (minimum
  and average) when legal.%
  }
  \label{tab:heuristic-dev}
\end{table}

One possible technique to selecting workgroup size is to tune
particular values for each targeted execution device. This approach is
sometimes adopted for cases with particularly high requirements for
performance on a single platform, so it produces an interesting
contrast to evaluating a machine learning approach, which attempts to
predict workgroup sizes for unseen platforms without the need for an
expensive exploration phase on the new platform.

Figure~\ref{fig:performances} shows the performance of workgroup sizes
relative to the oracle across scenarios grouped by: kernel, device,
and dataset. When grouped like this, a number of observations can
made. First is that not all of the kernels are sensitive to tuning
workgroup size to the same degree. The \emph{sobel} kernel has the
lowest median performance, indicating that it is the most profitable
to tune, while the \emph{threshold} kernel is the least
profitable. Similarly, the Intel i7-3820 is far less amenable to
tuning than the other devices, while the Intel i5-4570 is the most
sensitive to the workgroup size parameter. Such variances in the
distributions of workgroup sizes suggest that properties underlying
the architecture, kernel, and dataset all contribute towards the
proper selection of workgroup size.

To test the performance of a per-device heuristic for selecting
workgroup size, we group the scenarios by device, and compare the
relative performance of all workgroup sizes for each group of
scenarios. The most frequently optimal workgroup size $\bar{w}$ for
each device is selected, and the legality and performance of each
scenario using that workgroup size is evaluated.
Table~\ref{tab:heuristic-dev} shows the results of this evaluation.
The GTX 690 and GTX TITAN share the same $\bar{w}_{(64 \times 4)}$
value, while every other device has a unique optimum. The general case
performance of these per-device parameters is very good, although
legality is only assured for the GTX TITAN and AMD 7970 (which did not
refuse any parameters). However, the worst case performance of
per-device workgroup sizes is poor for all except the i7-3820 (which
is least sensitive to tuning), suggesting that device alone is not
capable of reliably informing the choice of workgroup size.


\cleardoublepage
\begin{figure}
  \begin{subfigure}[h]{\textwidth}
  \centering
  \includegraphics{img/performance_max_wgsize}
  \vspace{-1.5em} % Shrink vertical padding
  \caption{}
  \label{fig:performance-max-wgsize}
\end{subfigure}
\\
\begin{subfigure}[h]{.48\textwidth}
  \centering
  \includegraphics{img/performance_max_c}
  \vspace{-1.5em} % Shrink vertical padding
  \caption{}
  \label{fig:performance-wg-c}
\end{subfigure}
~%
\begin{subfigure}[h]{.48\textwidth}
  \centering
  \includegraphics{img/performance_max_r}
  \vspace{-1.5em} % Shrink vertical padding
  \caption{}
  \label{fig:performance-wg-r}
\end{subfigure}

  \caption[Workgroup size performances vs.\ size]{%
  Comparing workgroup performance relative to the oracle as function
  of: (\subref{fig:performance-max-wgsize})~maximum legal size,
  (\subref{fig:performance-wg-c})~number of columns, and
  (\subref{fig:performance-wg-r})~ number of rows. Each workgroup
  size is normalised to the maximum allowed for that scenario, $W_{\max}(s)$.%
  }
  \label{fig:performance-wgsizes}
\end{figure}
\begin{figure}
  \begin{subfigure}[h]{\textwidth}
  \centering
  \includegraphics{img/performance_kernels.pdf}
  \vspace{-1.5em} % Shrink vertical padding
  \caption{}
  \label{fig:performance-kernels}
\end{subfigure}
\\
\begin{subfigure}[h]{.48\textwidth}
  \centering
  \includegraphics{img/performance_devices.pdf}
  \vspace{-1.5em} % Shrink vertical padding
  \caption{}
  \label{fig:performance-devices}
\end{subfigure}
~%
\begin{subfigure}[h]{.48\textwidth}
  \centering
  \includegraphics{img/performance_datasets.pdf}
  \vspace{-1.5em} % Shrink vertical padding
  \caption{}
  \label{fig:performance-datasets}
\end{subfigure}

  \caption[Workgroup size performances across device, kernel, and dataset]{%
  Performance relative to the oracle of workgroup sizes across all
  test cases, grouped by: (\subref{fig:performance-kernels})~kernels,
  (\subref{fig:performance-devices})~devices, and
  (\subref{fig:performance-datasets})~datasets.%
  }
  \label{fig:performances}
\end{figure}


\subsection{Summary}

In this section we have explored the performance impact of the
workgroup size optimisation space. By comparing the relative
performance of an average of 629 workgroup sizes for each of 429
scenarios, the following conclusions can be drawn:

\begin{enumerate}
  \item The performance of a workgroup size for a particular scenario
  depends properties of the hardware, software, and dataset.
  \item The performance gap between the best and workgroup sizes for a
  particular combination of hardware, software, and dataset is up to
  $207.72\times$.
  \item Not all workgroup sizes are legal, and the space of legal
  workgroup sizes cannot statically be determined. Adaptive tuning of
  workgroup size is required to ensure reliable performance.
  \item Differing scenarios have wildly different optimal workgroup
  sizes, and the best performance can be achieved using static tuning
  is optimal for only 15\% of scenarios.
\end{enumerate}
%
% I believe this presents a compelling case for the development of an
% autotuner which can select the optimal workgroup size at runtime.
%
In the following section, we will evaluate the performance of OmniTune
for selecting workgroup sizes.


\section{Autotuning Workgroup Sizes}

In this section, we evaluate the performance of OmniTune for
predicting workgroup sizes of SkelCL skeletons using the prediction
techniques described in Section~\ref{sec:omnitune-ml}. For each
technique, we partition the experimental data into training and
testing sets, $S_{training} \subset S$ and
$S_{testing} = S - S_{training}$. A set of labelled training data
$D_{training}$ is derived from $S_{training}$, and the prediction
quality is testing using the validation set $D_{testing}$ derived from
$S_{training}$. We use multiple approaches to partitioning test data
to evaluate the prediction quality under different scenarios. The
processes for generating validation sets are:
%
\begin{itemize}
  \item 10-fold --- shuffle the set of all data and divide into 10
  validation sets, each containing 10\% of the data. This process is
  repeated for 10 rounds, resulting in 100 validations of 10
  permutations of the data.
  \item Synthetic --- divide the training data such that it consists
  solely of data collected from synthetic benchmarks, and use data
  collected from real-world benchmarks to test.
  \item Device --- partition the training data into $n$ sets, one for
  each device. Use $n-1$ sets for training, repeating until every
  partition has been used for testing once.
  \item Kernel --- partition the training data into $n$ sets, one for
  each kernel. Use $n-1$ sets for training, repeating until every
  partition has been used for testing once.
  \item Dataset --- partition the training data into $n$ sets, one for
  each type of dataset. Use $n-1$ sets for training, repeating until
  every partition has been used for testing once.
\end{itemize}
%
For each autotuning technique, the results of testing using the
different validation sets are reported separately. The autotuning
techniques evaluated are: using classifiers to predict the optimal
workgroup size of a stencil, with fallback strategies to handle
illegal predictions; using regressors to predict the runtime of a
stencil using different workgroup sizes, and selecting the legal
workgroup size which has the lowest predicted runtime; and using
regressors to predict the relative performance of workgroup sizes over
a baseline, and selecting the workgroup size which has the highest
predicted relative performance. We first describe the evaluation
strategies for each technique, before presenting experimental results
and analysis.


\subsection{Evaluating Classifiers}

The methodology for selecting workgroup size using classifiers is
described in Section~\ref{subsec:omnitune-ml-class}. Training data
consists of pairs of feature vectors $f(s)$ and oracle workgroup sizes
$\Omega(s)$:
%
\begin{equation}
  D_{training} = \left\{ (f(s),\Omega(s)) | s \in S_{training} \right\}
\end{equation}
%
Testing data are not labelled with oracle workgroup sizes:
%
\begin{equation}
  D_{testing} = \left\{ f(s) | s \in S_{testing} \right\}
\end{equation}
%
Each classifier is evaluated using the three different classification
techniques: \textsc{Baseline}, \textsc{Random}, and
\textsc{NearestNeighbour}, which differ in the way in which they
handle illegal predictions. Illegal predictions occur either because
the classifier has suggested a parameter which does not satisfy the
maximum workgroup size constraints $w < W_{\max}(s)$, or because the
workgroup size is refused by OpenCL $w \in W_{refused}(s)$. Workgroup
sizes are predicted for each scenario in the testing set, and the
quality of the predicted workgroup size is evaluated using the
following metrics:
%
\begin{itemize}
  \item accuracy (binary) --- whether the predicted workgroup size is
  the true oracle, $p(f(s)) = \Omega(s)$.
  \item validity (binary) --- whether the classifier predicted a
  workgroup size which satisfies the workgroup size constraints
  constraints, $p(f(s)) < W_{\max}(s)$.
  \item refused (binary) --- whether the classifier predicted a
  workgroup size which is refused, $p(f(s)) \in W_{refused}(s)$.
  \item performance (real) --- the relative performance of the predicted
  workgroup size relative to the oracle,
  $0 \le r(p(f(s)), \Omega(s)) \le 1$.
  \item speedups (real) --- the relative performance of the predicted
  workgroup size relative to the baseline workgroup size
  $w_{(4 \times 4)}$, and human expert workgroup size
  $w_{(32 \times 4)}$ (where applicable).
  \item time (real) --- the round trip time required to make the prediction,
  as measured by the OmniTune client. This includes classification
  time and inter-process communication overheads between the client
  and server.
\end{itemize}
%
The \emph{validty} and \emph{refused} metrics measure how often
fallback strategies are required to select a legal workgroup size
$w \in W_{legal}(s)$.


\subsection{Evaluating Regressors}

Sections~\ref{subsec:omnitune-ml-runtime}
and~\ref{subsec:omnitune-ml-speedup} describe methodologies for
selecting workgroup sizes by predicting program runtimes or relative
performance, respectively. The evaluation approach for both
methodologies is the same, only the training data differs. For
predicting runtimes, training data consists of feature vectors,
labelled with the mean observed runtime $t(s,w)$ for all legal
workgroup sizes:
%
\begin{equation}
  D_{training} = \bigcup_{\forall s \in S_{training}} \left\{ (f(s),t(s,w)) | w \in W_{legal}(s)
  \right\}
\end{equation}
For predicting speedups, the features vectors are labelled with
observed speedup over the baseline parameter $\bar{w}$ (see
Section~\ref{subsec:baseline}) for all legal workgroup sizes:
\begin{equation}
  D_{training} = \cup \left\{ (f(s),r(s,w,\bar{w})) | w \in W_{legal}(s)
  \right\} \forall s \in S_{training}
\end{equation}
%
Test data consists of unlabelled feature vectors:
%
\begin{equation}
  D_{testing} = \left\{ f(s) | s \in S_{testing} \right\}
\end{equation}
%
The quality of predicted workgroup sizes is evaluated using the
following metrics:
%
\begin{itemize}
  \item accuracy (binary) --- whether the predicted workgroup size is
  the true oracle, $p(f(s)) = \Omega(s)$.
  \item performance (real) --- the relative performance of the predicted
  workgroup size relative to the oracle,
  $0 \le r(p(f(s)), \Omega(s)) \le 1$.
  \item speedups (real) --- the relative performance of the predicted
  workgroup size relative to the baseline workgroup size
  $w_{(4 \times 4)}$, and human expert workgroup size
  $w_{(32 \times 4)}$ (where applicable).
  \item time (real) --- the round trip time required to make the
  prediction, as measured by the OmniTune client. This includes
  classification time and inter-process communication overheads
  between the client and server.
\end{itemize}
%
Unlike with classifiers, the process of selecting workgroup sizes
using regressors is resistant to refused parameters, so no fallback
strategies are required, and the \emph{validity} and \emph{refused}
metrics are not used.


\subsection{Results and Analysis}

The purpose of this evaluation is to test the effectiveness of machine
learning-enabled autotuning for predicting workgroup sizes of SkelCL
stencils codes. First, we consider the prediction performance of
classifiers.


\cleardoublepage
\begin{figure}
  \centering
  \includegraphics{img/classification-syn-real}
  \caption[Classification results using synthetic benchmarks]{%
  Classification results for synthetic benchmarks. Each classifier is
  trained on data from synthetic stencils, and tested for prediction
  quality using data from 6 real world benchmarks.%
  }
  \label{fig:class-syn}
\end{figure}
\newpage
\begin{figure}
  \centering
  \includegraphics{img/classification-arch}
  \caption[Classification results using cross-device evaluation]{%
  Classification results of cross-device evaluation. Each classifier
  is trained using data from $n-1$ devices, and tested for prediction
  quality using data for the $n^{th}$ device.%
  }
  \label{fig:class-arch}
\end{figure}
\begin{figure}
  \centering
  \begin{subfigure}[h]{.45\textwidth}
    \centering
    \includegraphics{img/runtime-class-xval}
    \caption{}
    \label{fig:runtime-class-xval}
  \end{subfigure}
  ~%
  \begin{subfigure}[h]{.45\textwidth}
    \centering
    \includegraphics{img/speedup-class-xval}
    \caption{}
    \label{fig:speedup-class-xval}
  \end{subfigure}
  \caption[Autotuning performance using regressors]{%
  Evaluating the effectiveness of classification using regressors, by
  predicting: (\subref{fig:runtime-class-xval}) the workgroup size
  with the minimal runtime, and (\subref{fig:speedup-class-xval}) the
  workgroup size with the greatest speedup over a baseline.%
  }
  \label{fig:regression-class}
\end{figure}

With the exception of the ZeroR, which predicts \emph{only} the
baseline workgroup size $w_{\left( 4 \times 4 \right)}$, the
classifiers achieve good speedups over the baseline. Average
classification speedups across all validation sets range between
$4.61\times$ and $5.05\times$. Figures~\ref{fig:class-syn}
and~\ref{fig:class-arch} show a summary of results using 10-fold
cross-validation and cross-device validation, respectively. The
highest average speedup is achieved by SMO, and the lowest by Naive
Bayes. The difference between average speedups is not significant
between the types of classifier, with the exception of SimpleLogistic,
which performs poorly when trained with synthetic benchmarks and
tested against real-world programs. This suggests the model
over-fitting to features of the synthetic benchmarks which are not
shared by the real-world
tests.

\begin{figure}
  \centering
  \includegraphics{img/fallback_speedups}
  \caption[Comparison of fallback handler speedups]{%
  Comparison of fallback handlers, showing the speedup over baseline
  parameter for all test cases where a classifier predicted an illegal
  workgroup size.%
  }
  \label{fig:fallback-speedups}
\end{figure}

By isolating the test cases where classifiers predicted an illegal or
refused parameter, we can directly compare the relative effectiveness
of each fallback handler. The fallback handler with the best average
case performance is \textsc{NearestNeighbour}, with an average speedup
across all classifiers and validation sets of $5.26\times$. The
speedup of \textsc{Random} fallback handler is $3.69\times$, and
$1.0\times$ for \textsc{Baseline}. Figure~\ref{fig:fallback-speedups}
plots the speedups of each fallback handler for all of these isolated
test cases. Interestingly, both the lowest and highest speedups are
achieved by the \textsc{Random} fallback handler, since it essentially
performs a random exploration of the optimisation space. However, the
\textsc{NearestNeighbour} fallback handler provides consistently
greater speedups for the majority of test cases, indicating that it
successfully exploits the structure of the optimisation spaces.

Figures~\ref{fig:runtime-class-xval} and ~\ref{fig:speedup-class-xval}
show a summary of results for classification using regressors to
predict program runtimes and speedups, respectively. Of the two
regression techniques, predicting the \emph{speedup} of workgroup
sizes is much more successful than predicting the \emph{runtime}. This
is most likely caused by the inherent difficulty in predicting the
runtime of arbitrary programs, where dynamic factors such as flow
control and loop bounds are not captured by the kernel features used
in OmniTune, which instead use simple static static instruction counts
and densities. The average speedup achieved by predicting runtimes is
$4.14\times$. For predicting speedups, the average is $5.57\times$.
Tables~\ref{tab:class}, \ref{tab:runtime-class},
and~\ref{tab:speedup-class} show mean performances and speedups for:
J48 classifier using the \textsc{NearestNeighour} fallback strategy,
classification using runtime regression, and classification using
speedup regression, respectively.

If we eliminate the 2.6\% of test cases for which the workgroup size
of $w_{(32 \times 4)}$ is illegal, we can compare the performance of
OmniTune directly against the human expert chosen workgroup
size. Figure~\ref{fig:speedup-distributions} compares the speedups of
all such validation instances over the human expert parameter, for
each autotuning technique. The speedup distributions show consistent
classification results for the five classification techniques, with
the speedup at the lower quartile for all classifiers being
$\ge 1.0\times$. The IQR for all classifiers is $< 0.5$, but there are
outliers with speedups both well below $1.0\times$ and well above
$2.0\times$. In contrast, the speedups achieved using runtime
regression have a lower range, but also a lower median and a larger
IQR. Clearly, runtime regression is the least effective of the
evaluated autotuning techniques. Speedup regression is more
successful, with the highest median speedup of all the
techniques. However, it also has a large IQR and the lower quartile
has a speedup value well below 1, meaning that for more than 25\% of
test instances, the workgroup size selected did not perform as well as
the human expert selected workgroup size.

% The lower average speedup attained by speedup regression over the
% J48 classifier belies the fact that the \emph{median} speedup is
% much greater, at $1.33 \times$.


\begin{figure}
  \centering
  \includegraphics{img/speedup-distributions}
  \caption[Speedup results over human expert]{%
  Distributions of speedups over \emph{human expert}, ignoring cases
  where human expert prediction is invalid. Classifiers are using
  \textsc{NearestNeighbour} fallback handlers. The speedup axis is
  fixed to the range 0--2.5 to highlight the IQRs, which results in
  some outliers > 2.5 being clipped.%
  }
  \label{fig:speedup-distributions}
\end{figure}






\begin{table}
  \scriptsize
  \centering
  \rowcolors{2}{white}{gray!25}
  \begin{tabular}{llll}
  \toprule
  Job & Performance & Speedup & Human Expert \\
  \midrule
  10-fold & 92\% &       $5.65\times$ &       $1.26\times$ \\
  Synthetic & 92\% &       $4.79\times$ &       $1.13\times$ \\
  Device & 85\% &       $5.23\times$ &       $1.17\times$ \\
  Kernel & 89\% &       $5.43\times$ &       $1.21\times$ \\
  Dataset & 91\% &       $5.63\times$ &       $1.25\times$ \\
  \textbf{Average} &  \textbf{90\%} &  $\bm{5.45\times}$ &  $\bm{1.22\times}$ \\
  \bottomrule
\end{tabular}


  \caption{Validation results for J48 and \textsc{NearestNeighbour}
  classification.}
  \label{tab:class}
\end{table}
\begin{table}
  \scriptsize
  \centering
  \rowcolors{2}{white}{gray!25}
  \begin{tabular}{llll}
  \toprule
  Job & Performance & Speedup & Human Expert \\
  \midrule
  10-fold & 68\% &       $4.13\times$ &       $0.88\times$ \\
  Synthetic & 78\% &       $3.81\times$ &       $1.06\times$ \\
  Device & 69\% &       $3.89\times$ &       $0.97\times$ \\
  Kernel & 74\% &       $4.36\times$ &       $1.04\times$ \\
  Dataset & 72\% &       $4.33\times$ &       $0.98\times$ \\
  \textbf{Average} &  \textbf{70\%} &  $\bm{4.14\times}$ &  $\bm{0.92\times}$ \\
  \bottomrule
\end{tabular}


  \caption{Validation results for runtime regression.}
  \label{tab:runtime-class}
\end{table}
\begin{table}
  \scriptsize
  \centering
  \rowcolors{2}{white}{gray!25}
  \begin{tabular}{llll}
  \toprule
  Job & Performance & Speedup & Human Expert \\
  \midrule
  10-fold & 89\% &       $5.67\times$ &       $1.10\times$ \\
  Synthetic & 86\% &       $4.48\times$ &       $1.19\times$ \\
  Device & 85\% &       $5.18\times$ &       $1.15\times$ \\
  Kernel & 88\% &       $5.38\times$ &       $1.15\times$ \\
  Dataset & 88\% &       $5.53\times$ &       $1.13\times$ \\
  \textbf{Average} &  \textbf{89\%} &  $\bm{5.57\times}$ &  $\bm{1.12\times}$ \\
  \bottomrule
\end{tabular}


  \caption{Validation results for speedup regression.}
  \label{tab:speedup-class}
\end{table}


% \subsection{Visualising Prediction Errors}

\begin{figure}
  \centering
  \begin{subfigure}[t]{0.45\textwidth}
    \centering
    \includegraphics{img/heatmap_1}
    \vspace{-1.5em} % Shrink vertical padding
    \caption{}
    \label{fig:class-hmaps-1}
  \end{subfigure}
  ~%
  \begin{subfigure}[t]{0.45\textwidth}
    \centering
    \includegraphics{img/heatmap_2}
    \vspace{-1.5em} % Shrink vertical padding
    \caption{}
    \label{fig:class-hmaps-2}
  \end{subfigure}
  \\
  \begin{subfigure}[t]{0.45\textwidth}
    \centering
    \includegraphics{img/heatmap_3}
    \vspace{-1.5em} % Shrink vertical padding
    \caption{}
    \label{fig:class-hmaps-3}
  \end{subfigure}
  ~%
  \begin{subfigure}[t]{0.45\textwidth}
    \centering
    \includegraphics{img/heatmap_5}
    \vspace{-1.5em} % Shrink vertical padding
    \caption{}
    \label{fig:class-hmaps-4}
  \end{subfigure}
  \\
  \begin{subfigure}[t]{0.45\textwidth}
    \centering
    \includegraphics{img/reg_runtime_heatmap}
    \vspace{-1.5em} % Shrink vertical padding
    \caption{}
    \label{fig:class-hmaps-5}
  \end{subfigure}
  ~%
  \begin{subfigure}[t]{0.45\textwidth}
    \centering
    \includegraphics{img/reg_speedup_heatmap}
    \vspace{-1.5em} % Shrink vertical padding
    \caption{}
    \label{fig:class-hmaps-6}
  \end{subfigure}
  \caption[Classification error heatmaps]{%
  Heatmaps of classification errors for 10-fold cross-validation,
  showing a subset of the optimisation space. The shading in each
  cells indicates if it is predicted less frequently (blue), ore more
  frequently (red) than it is optimal. Colour gradients are normalised
  across plots.%
  }
  \label{fig:class-hmaps}
\end{figure}


The prediction costs using regression are significantly greater than
using classifiers. This is because, while a classifier makes a single
prediction, the number of predictions required of a regressor grows
with the size of $W_{\max}(s)$, since classification with regression
requires making predictions for all
$w \in \left\{ w | w < W_{\max}(s) \right\}$. The fastest classifier
is J48, due to the it's simplicity (it can be implemented as a
sequence of nested \texttt{if}/\texttt{else} statements).

Figure~\ref{fig:class-hmaps} visualises the classification errors of
each of the autotuning techniques. It shows that while the performance
of all of the classifiers is comparable, the distribution of
predictions is not. Only the NaiveBayes and RandomForest classifiers
predicted the human expert selected workgroup size of
$w_{(32 \times 4)}$ as frequently, or more frequently, than it was
optimal. The two regression techniques were the least accurate of all
of the autotuning techniques.


\subsection{Summary}

From an evaluation of 17 different autotuning techniques using 5
different types of validation sets, the following conclusions about
autotuning performance can be drawn:
%
\begin{itemize}
  \item In the case of classifiers predicting illegal workgroup sizes,
  the best fallback strategy is to select the closest legal workgroup
  size.
  \item The performance of predicted workgroup sizes for unseen devices
  is within 8\% of the performance for known devices.
  \item Predicting the \emph{runtime} of stencils is the least effective
  of the evaluated autotuning techniques, achieving an average of only
  68\% of the available performance.
  \item Predicting the \emph{speedup} of workgroup sizes provides the
  highest median speedup, but more frequently predicts a poorly
  performing workgroup size then the classifiers.
  \item Classification using regression costs an order of magnitude more
  time than using classifiers. The J48 classifier has the lowest
  overhead.
\end{itemize}
