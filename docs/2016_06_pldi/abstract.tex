%
% Submit a single PDF file on the HotCRP PLDI16 Posters submission
% system with the following contents (at least 3, at most 5 pages):
%
% * An abstract of at most two pages (preferably just one full page)
% describing the research you want to present including how it is
% relevant to PLDI
%
% * A one page draft of your poster (shrink to fit in a single 8.5inx11
% in page) in portrait layout
%
% * A letter of support (at most two pages, preferably one) from your
% academic adviser or PhD supervisor for including your poster in the
% student poster session at PLDI16
%
% Contact the PLDI16 Poster Chair with questions.
%
\documentclass[sigconf,preprint]{acmart}

%%% The following is specific to ISSTA'18 and the paper
%%% 'Compiler Fuzzing through Deep Learning'
%%% by Chris Cummins, Pavlos Petoumenos, Alastair Murray, and Hugh Leather.
%%%
\setcopyright{acmlicensed}
\acmPrice{15.00}
\acmDOI{10.1145/3213846.3213848}
\acmYear{2018}
\copyrightyear{2018}
\acmISBN{978-1-4503-5699-2/18/07}
\acmConference[ISSTA'18]{27th ACM SIGSOFT International Symposium on Software Testing and Analysis}{July 16--21, 2018}{Amsterdam, Netherlands}

\begin{CCSXML}
  <ccs2012>
  <concept>
  <concept_id>10011007.10011074.10011099.10011102.10011103</concept_id>
  <concept_desc>Software and its engineering~Software testing and debugging</concept_desc>
  <concept_significance>500</concept_significance>
  </concept>
  </ccs2012>
\end{CCSXML}

\ccsdesc[500]{Software and its engineering~Software testing and debugging}

\keywords{Deep Learning; Differential Testing; Compiler Fuzzing.}

\usepackage[utf8]{inputenc}

\usepackage[normalem]{ulem}

% Start of 'ignore natbib' hack
\let\bibhang\relax
\let\citename\relax
\let\bibfont\relax
\let\Citeauthor\relax
\expandafter\let\csname ver@natbib.sty\endcsname\relax
% End of 'ignore natbib' hack

\usepackage{graphicx}

% Enable \subfloat{} command.
\usepackage{subfig}

% Tables.
\usepackage{booktabs}
\usepackage{tabularx}
\usepackage{hhline}
\usepackage{xspace}

% Define column types L, C, R with known text justification and fixed widths:
\usepackage{array}
\newcolumntype{L}[1]{>{\raggedright\let\newline\\\arraybackslash\hspace{0pt}}m{#1}}
\newcolumntype{C}[1]{>{\centering\let\newline\\\arraybackslash\hspace{0pt}}m{#1}}
\newcolumntype{R}[1]{>{\raggedleft\let\newline\\\arraybackslash\hspace{0pt}}m{#1}}

% Source code listings.
\usepackage{courier}
\usepackage{listings,lstautogobble}
\lstset{%
basicstyle=\scriptsize\ttfamily,%
numbers=left,%
xleftmargin=1em,
framexleftmargin=2.5em,
framexrightmargin=-2em,
escapeinside={@|}{|@},
frame=b,
breaklines=true,
postbreak=\raisebox{0ex}[0ex][0ex]{\ensuremath{\color{red}\hookrightarrow\space}},% red arrow at line breaks
captionpos=b,
autogobble=true % indent listing based on first line
}

% OpenCL listings
% From: http://gpumodeling.blogspot.com/2011/06/opencl-programs-in-latex-listings.html
\lstdefinelanguage[OpenCL]{C}[ANSI]{C}
{morekeywords={__kernel,kernel,__local,local,__global,global,%
__constant,constant,__private,private,%
__read_only,read_only,__write_only,write_only,%
char2,char3,char4,char8,char16,%
uchar2,uchar3,uchar4,uchar8,uchar16,%
short2,short3,short4,short8,short16,%
ushort2,ushort3,ushort4,ushort8,ushort16,%
int2,int3,int4,int8,int16,%
uint2,uint3,uint4,uint8,uint16,%
long2,long3,long4,long8,long16,%
ulong2,ulong3,ulong4,ulong8,ulong16,%
float2,float3,float4,float8,float16,%
image2d_t,image3d_t,sampler_t,event_t,size_t,%
bool2,bool3,bool4,bool8,bool16,%
half2,half3,half4,half8,half16,%
quad,quad2,quad3,quad4,quad8,quad16,%
complex,imaginary,barrier},%
}%

% A \ceil{} operator.
\usepackage{mathtools}
\DeclarePairedDelimiter{\ceil}{\lceil}{\rceil}

% A \cmark and \xmark symbol for testcase outcomes.
\usepackage{amssymb}
\usepackage{pifont}
\usepackage{multirow}
\newcommand{\cmark}{\ding{51}}
\newcommand{\xmark}{\ding{55}}

% Test case outcome classifications.
\newcommand\bc{\textbf{bc}\xspace}
\newcommand\bto{\textbf{bto}\xspace}
\newcommand\abf{\textbf{abf}\xspace}
\newcommand\arc{\textbf{arc}\xspace}
\newcommand\awo{\textbf{awo}\xspace}

% Balanced columns on the last page.
\usepackage{flushend}


\usepackage{pdfpages}

\begin{document}

  \title{Autotuning OpenCL Workgroup Sizes}

  \author{Chris Cummins, University of Edinburgh}

  \maketitle

  The physical limitations of microprocessor design have forced the
  industry towards increasingly heterogeneous designs to extract
  performance, with an an increasing pressure to offload traditionally
  CPU based workloads to the GPU. This trend has not been matched with
  adequate software tools; the popular languages OpenCL and CUDA provide
  a very low level model with little abstraction above the
  hardware. Programming at this level requires expert knowledge of both
  the domain and the target hardware, and achieving performance requires
  laborious hand tuning of each program. This has led to a growing
  disparity between the availability of parallelism in modern hardware,
  and the ability for application developers to exploit it.

  The goal of this work is to bring the performance of hand tuned
  heterogeneous code to high level programming, by incorporating
  autotuning into \textit{Algorithmic Skeletons}. Algorithmic Skeletons
  simplify parallel programming by providing reusable, high-level,
  patterns of computation. However, achieving performant skeleton
  implementations is a difficult task; skeleton authors must attempt to
  anticipate and tune for a wide range of architectures and use
  cases. This results in implementations that target the general case
  and cannot provide the performance advantages that are gained from
  tuning low level optimization parameters for individual programs and
  architectures. Autotuning combined with machine learning offers
  promising performance benefits by tailoring parameter values to
  individual cases, but the high cost of training and the ad-hoc nature
  of autotuning tools limits the practicality of autotuning for real
  world programming. We believe that performing autotuning at the level
  of the skeleton library can overcome these issues.

  In this work, we present \textit{OmniTune} --- an extensible and
  distributed framework for autotuning optimization parameters in
  algorithmic skeletons at runtime. OmniTune enables a collaborative
  approach to performance tuning, in which machine learning training
  data is shared across a network of cooperating systems, amortizing the
  cost of exploring the optimization space. We demonstrate the
  practicality of OmniTune by autotuning the OpenCL workgroup size of
  stencil skeletons in SkelCL. SkelCL is an Algorithmic Skeleton
  framework which abstracts the complexities of OpenCL programming,
  exposing a set of data parallel skeletons for high level heterogeneous
  programming in C++. Selecting an appropriate OpenCL workgroup size is
  critical for the performance of programs, and requires knowledge of
  the underlying hardware, the data being operated on, and the program
  implementation. Our autotuning approach employs the novel application
  of linear regressors for classification of workgroup size, extracting
  102 features at runtime describing the program, device, and dataset,
  and predicting optimal workgroup sizes based on training data
  collected using synthetically generated stencil benchmarks.

  In an empirical study of 429 combinations of programs, architectures,
  and datasets, we find that OmniTune provides a median $3.79\times$
  speedup over the best possible fixed workgroup size, achieving 94\% of
  the maximum performance. Our results demonstrate that autotuning at
  the skeletal level --- when combined with sophisticated machine
  learning techniques --- can raise the performance above that of human
  experts, without requiring any effort from the user. By introducing
  OmniTune and demonstrating its practical utility, we hope to
  contribute to the increasing uptake of autotuning techniques into
  tools and languages for high level programming of heterogeneous
  systems.

  \newpage

  \includepdf{poster.pdf}

  \newpage

  \includepdf{hugh-letter.pdf}


\end{document}
