\documentclass[sigconf,preprint]{acmart}

%%% The following is specific to ISSTA'18 and the paper
%%% 'Compiler Fuzzing through Deep Learning'
%%% by Chris Cummins, Pavlos Petoumenos, Alastair Murray, and Hugh Leather.
%%%
\setcopyright{acmlicensed}
\acmPrice{15.00}
\acmDOI{10.1145/3213846.3213848}
\acmYear{2018}
\copyrightyear{2018}
\acmISBN{978-1-4503-5699-2/18/07}
\acmConference[ISSTA'18]{27th ACM SIGSOFT International Symposium on Software Testing and Analysis}{July 16--21, 2018}{Amsterdam, Netherlands}

\begin{CCSXML}
  <ccs2012>
  <concept>
  <concept_id>10011007.10011074.10011099.10011102.10011103</concept_id>
  <concept_desc>Software and its engineering~Software testing and debugging</concept_desc>
  <concept_significance>500</concept_significance>
  </concept>
  </ccs2012>
\end{CCSXML}

\ccsdesc[500]{Software and its engineering~Software testing and debugging}

\keywords{Deep Learning; Differential Testing; Compiler Fuzzing.}

\usepackage[utf8]{inputenc}

\usepackage[normalem]{ulem}

% Start of 'ignore natbib' hack
\let\bibhang\relax
\let\citename\relax
\let\bibfont\relax
\let\Citeauthor\relax
\expandafter\let\csname ver@natbib.sty\endcsname\relax
% End of 'ignore natbib' hack

\usepackage{graphicx}

% Enable \subfloat{} command.
\usepackage{subfig}

% Tables.
\usepackage{booktabs}
\usepackage{tabularx}
\usepackage{hhline}
\usepackage{xspace}

% Define column types L, C, R with known text justification and fixed widths:
\usepackage{array}
\newcolumntype{L}[1]{>{\raggedright\let\newline\\\arraybackslash\hspace{0pt}}m{#1}}
\newcolumntype{C}[1]{>{\centering\let\newline\\\arraybackslash\hspace{0pt}}m{#1}}
\newcolumntype{R}[1]{>{\raggedleft\let\newline\\\arraybackslash\hspace{0pt}}m{#1}}

% Source code listings.
\usepackage{courier}
\usepackage{listings,lstautogobble}
\lstset{%
basicstyle=\scriptsize\ttfamily,%
numbers=left,%
xleftmargin=1em,
framexleftmargin=2.5em,
framexrightmargin=-2em,
escapeinside={@|}{|@},
frame=b,
breaklines=true,
postbreak=\raisebox{0ex}[0ex][0ex]{\ensuremath{\color{red}\hookrightarrow\space}},% red arrow at line breaks
captionpos=b,
autogobble=true % indent listing based on first line
}

% OpenCL listings
% From: http://gpumodeling.blogspot.com/2011/06/opencl-programs-in-latex-listings.html
\lstdefinelanguage[OpenCL]{C}[ANSI]{C}
{morekeywords={__kernel,kernel,__local,local,__global,global,%
__constant,constant,__private,private,%
__read_only,read_only,__write_only,write_only,%
char2,char3,char4,char8,char16,%
uchar2,uchar3,uchar4,uchar8,uchar16,%
short2,short3,short4,short8,short16,%
ushort2,ushort3,ushort4,ushort8,ushort16,%
int2,int3,int4,int8,int16,%
uint2,uint3,uint4,uint8,uint16,%
long2,long3,long4,long8,long16,%
ulong2,ulong3,ulong4,ulong8,ulong16,%
float2,float3,float4,float8,float16,%
image2d_t,image3d_t,sampler_t,event_t,size_t,%
bool2,bool3,bool4,bool8,bool16,%
half2,half3,half4,half8,half16,%
quad,quad2,quad3,quad4,quad8,quad16,%
complex,imaginary,barrier},%
}%

% A \ceil{} operator.
\usepackage{mathtools}
\DeclarePairedDelimiter{\ceil}{\lceil}{\rceil}

% A \cmark and \xmark symbol for testcase outcomes.
\usepackage{amssymb}
\usepackage{pifont}
\usepackage{multirow}
\newcommand{\cmark}{\ding{51}}
\newcommand{\xmark}{\ding{55}}

% Test case outcome classifications.
\newcommand\bc{\textbf{bc}\xspace}
\newcommand\bto{\textbf{bto}\xspace}
\newcommand\abf{\textbf{abf}\xspace}
\newcommand\arc{\textbf{arc}\xspace}
\newcommand\awo{\textbf{awo}\xspace}

% Balanced columns on the last page.
\usepackage{flushend}
